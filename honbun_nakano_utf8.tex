\documentclass[12pt]{mthesis_utf8}\usepackage{graphicx}
\usepackage{latexsym}

\newtheorem{lem}{補題}
\newtheorem{prop}{命題}[section]
\newtheorem{prf}{証明}
\newtheorem{theo}{定理}
\newtheorem{Definition}{定義}


% タイトル
% 自分で改行位置を指定する場合は,'\newline' を用いる
\title{アクセス環境によるインターネット通信品質に関する研究}
\etitle{Research on Internet Communication Quality by Access Environment}


%\mti{○○○○に関する研究}
% 指導教官名(敬称は『教授』が自動的に付く)
\supervisor{高野 知佐}
\esupervisor{Chisa Takano}

% 入学年度(『20XX 年度入学』となる)
\admdate{2023}

% 学籍番号
\regnum{2366010}

% 著者名
\author{中野 龍太朗}
\eauthor{Ryutaro Nakano}

% 提出日(『…提出』となる)
\submission{2024 年 1月 24日}

% ヘッダを付ける場合はコメントアウトする
\usehead

% ASCII-pTeX の場合はコメントアウトする
\asciitex

% ドラフト・モード
% フローティング・オブジェクトを章末に追いやり,本文のみの
% 文章量を数え易くする(自動計算は出来ない).
\draftmode

\setlength{\baselineskip}{12pt}
%\affiliate[所属ラベル]{和文勤務先\\ 連絡先住所}{英文勤務先\\ 英文連絡先住所}

\begin{document}
\maketitle
\begin{abstract}
ここには,修士論文の概要を書く.(A4 1ページ程度)
\end{abstract}

\begin{eabstract}
Write abstract of your thesis here within a page in A4 size.
\end{eabstract}


%%
%%%% 1------------------------------------------------------------
%%
%%%%%%%%%%%%%%%%%%%%%%%%%%%%%%%%%%%%%%%%%%%%%%%%%%%%%%%%%%%%%%%%%%%%%%%%%%

\chapter{序論}
\section{背景と目的}


\section{本論文の構成}

%%
%%%% 2------------------------------------------------------------
%%
%%%%%%%%%%%%%%%%%%%%%%%%%%%%%%%%%%%%%%%%%%%%%%%%%%%%%%%%%%%%%%%%%%%%%%%%%%

\chapter{関連研究}

\chapter{アクセス環境の影響の調査}
\section{調査に使用するデータ}
\section{スループットの調査}
\section{RTTの調査}
\section{調査のまとめ}

\chapter{可視化システム}
\section{システムの概要}
\section{システムの評価}
\section{まとめ}

\chapter{まとめ}

%\setlength{\baselineskip}{20pt}

%%%%%%%%%%%%%%%%%%%%%%%%%%%%%%%%%%%%%%%%%%%%%%%%%%%%%%%%%%%%%%%%%%%%%%%%%%


\begin{acknowledgment}
日頃より熱心に御指導下さった本学情報科研究科の△△ △△教授,ならびに,
本論文の細部にまで目を通し,貴重な助言を下さった□□ □□准教授と×× ××教授
に心より感謝致します.

また,御討論頂いた本学情報科学研究科○○○研究室の諸氏に感謝致します.
\end{acknowledgment}

\bibliographystyle{abbrv}
%
\begin{thebibliography}{99}
\bibitem {dews2007}
市大太郎,“論文題目”,学会名,ページ数,年.
%
\end{thebibliography}
%
\chapter*{業績リスト}

\section*{学会誌発表論文}
\begin{enumerate}
%
\item 市大太郎,“論文題目”,学会名,ページ数,年.
%
\end{enumerate}

\section*{国際会議発表論文}
\begin{enumerate}
%
\item 市大太郎,“論文題目”,学会名,ページ数,年.
%
\end{enumerate}

\section*{国内研究会発表論文}
\begin{enumerate}
%
\item 市大太郎,“論文題目”,学会名,ページ数,年.
%
\end{enumerate}
%
\end{document}
