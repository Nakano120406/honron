\chapter{関連研究・技術}
\label{chap:relatedwork}

\section{インターネット通信品質の調査}
ユーザーがインターネット通信品質を調査する方法に,Web ベースの計測サイトを利用するのが挙げられる.代表的な例として,Netflix が提供する\footnote{Fast.com},Google が提供する Speedtest などが広く使われており,これらのサイトではスループットや通信遅延,パケットロス率などの計測が行われる.みんなのネット回線速度\cite{minsoku}では計測に加えて都道府県やプロバイダ毎のスループットと遅延,遅延の揺らぎの平均値やランキングが掲載され,他のユーザーの計測結果と比較することができる.
これらのようなスピードテストの計測ログを利用してインターネットの品質の調査や評価に関する研究もされている.\cite{yasnyan}では,iNonius Speed Test\cite{iNonius}の全てのユーザーの計測ログを用いて,国内のインターネット通信の品質をIPv4とIPv6の両者で比較し,IPv6の利用が増加する中での品質の変化を調査している.また,\cite{reisan}では,一人のユーザーによるiNonius Speed Testの計測結果を用いて,通信品質について考察する手法を提案されている.
%しかし,これらの分析にはユーザーのアクセス環境を考慮していない場合が多く,考慮されている場合でもユーザーの自己申告であったり,IP アドレスからキャリア回線の判別のみされているなど限定的な場合が多い.

ほかにも,各IXが通過するトラフィックを基に統計情報を提供\cite{IIR}したり,総務省が日本のIXの協力のもと固定系ブロードバンドサービスで交換されているトラフィック量の推定値を公表している\cite{soumusho}など,インターネットの通信品質に関する調査や評価は様々な視点から行われている.

\section{Network Information API}
サーバー側でユーザーのアクセス環境を取得する方法は少ない.1 つの方法として Network Information API がある.これはJavascriptを使用してWebブラウザがネットワーク接続に関する情報を取得するためのAPIである.取得できる情報は\cref{tab:networkinfo}の通りである.接続タイプはユーザーが使用している端末から最初のワンホップに関する情報で,Wi-Fi,セルラー,イーサネットなどの接続タイプだけでなく,それらのバージョンも取得することができる.また,スループットと RTT の計測を行い,それらの結果から推定される接続タイプを取得することもできる.
本研究でこのAPIを使用したアクセス環境の推定に使える可能性がある.しかし,このAPIを使用した推定の精度に関する調査が行われていないため未知数であることから,本研究では使用していない.このAPIの有効性が示されれば,今後の研究で活用できる可能性がある.

\begin{table}[htbp]
    \centering
    \caption{Network Information API で取得できる情報}
    \begin{tabular}{cc}
        \hline
        プロパティ & 説明 \\
        \hline \hline
        ConnectionType & ヘッダー情報から取得したユーザーの接続タイプ \\
        effectiveType & スループットとRTTの実行速度から推定される接続タイプ \\
        downlinkMax & 接続タイプに応じた最大のダウンリンク速度 \\
        downlink & ユーザーのダウンロードの有効帯域幅 \\
        rtt & ユーザの実効ラウンドトリップタイム \\
        \hline
    \end{tabular}
    \label{tab:networkinfo}
\end{table}
\FloatBarrier
